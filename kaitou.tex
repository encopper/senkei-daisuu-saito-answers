\documentclass[dvipdfmx,uplatex,11pt]{jsarticle}
%
\usepackage[dvipdfmx]{graphicx}
\usepackage{amsmath,amssymb,amsthm}
\usepackage{enumitem}
\usepackage{wrapfig}
\usepackage{bm}
\usepackage{ascmac}
\setcounter{tocdepth}{2}
\usepackage{geometry}
\usepackage{framed}
\usepackage{latexsym}
\usepackage[dvipdfmx]{hyperref}
\usepackage{pxjahyper}
\hypersetup{
setpagesize=false,
 bookmarksnumbered=true,%
 bookmarksopen=true,%
 colorlinks=true,%
 linkcolor=blue,
 citecolor=red,
}
\everymath{\displaystyle}
%
\theoremstyle{definition}
\newtheorem{theo}{定理}[section]
\newtheorem{prop}{命題}[section]
\newtheorem{defi}{定義}[section]
\newtheorem{lemm}{補題}[section]
\newtheorem{exam}{具体例}[section]
%
\geometry{left=10mm,right=10mm,top=5mm,bottom=10mm}
%
\title{線型代数入門:解答集}
\author{編集者:}
\date{最終更新日:\today}
%
\begin{document}
\maketitle
\tableofcontents
\newpage
%
%
%
\section{第1章:平面および空間のベクトル(解答)}
%
\subsection{p5}

問1:

\begin{leftbar}
	\begin{proof}
		線分PQの中点をMとする.このとき,
		\begin{align*}
		\overrightarrow{\mathrm{OM}} & = \overrightarrow{\mathrm{OP}} + \overrightarrow{\mathrm{PM}} \\
		& = \bm{a} + \frac{\bm{b}-\bm{a}}{2} \\
		& = \frac{\bm{a}+\bm{b}}{2}
		\end{align*}
		である.
	\end{proof}
\end{leftbar}

問2:
\begin{leftbar}
	\begin{proof}
		三角形PQRの重心をG,PQの中点をNとする.Gは線分RNを$2:1$に内分する点なので,
		\begin{align*}
			\overrightarrow{\mathrm{OG}} &= \overrightarrow{\mathrm{OR}} + \frac{2}{3} \overrightarrow{\mathrm{RN}} \\
			& = \bm{c}+ \frac{2}{3} \left(\frac{\bm{a}+\bm{b}}{2}-\bm{c} \right)\\
			& = \frac{\bm{a}+\bm{b}+\bm{c}}{3}
		\end{align*}
		である.
	\end{proof}
\end{leftbar}

\newpage

\subsection{p7}

問:
\begin{leftbar}
    求めるベクトルを,$\bm{x}=(x,y,z)$~($x^2+y^2 +z^2=1$)とおく.
    このとき,内積の定義により,
    \begin{gather*}
        \bm{x} \cdot \left(
            \begin{array}{c}
                1 \\
                1 \\
                1
            \end{array}
        \right)
        =x+y+z= 1 \cdot \sqrt{3} \cdot \cos \frac{\pi}{6} =\frac{3}{2} \\
        \bm{x} \cdot \left(
            \begin{array}{c}
                1 \\
                1 \\
                4
            \end{array}
        \right)
        =x+y+4z= 1 \cdot 3\sqrt{2} \cdot \cos \frac{\pi}{4} =3
        \end{gather*}
        これらの式から,
        \begin{gather*}
        \left(
            \begin{array}{c}
            x \\
            y \\
            z  
    \end{array}
        \right)
    =
            \left(
                \begin{array}{c}
                \frac{2 \pm \sqrt{2}}{4} \\
                \frac{2 \mp \sqrt{2}}{4} \\
                \frac{1}{2} 
    \end{array}
            \right)\quad (\text{複合同順})
    \end{gather*}
        である.
    \end{leftbar}

問: 
\begin{leftbar}
	[1.4]の結果を利用する.\par 
	求める三角形の面積を$S$,$\bm{a}=\overrightarrow{\mathrm{P_1 P_2}}=(x_2-x_1,y_2-y_1,z_2-z_1)$,$\bm{b}=\overrightarrow{\mathrm{P_1 P_3}}=(x_3-x_1,y_3-y_1,z_3-z_1)$とおく,
	このとき,
	\begin{align*}
		S & = \frac{1}{2} \sqrt{\|\overrightarrow{\mathrm{P_1 P_2}}\|^2 \|\overrightarrow{\mathrm{P_1 P_3}}\|^2 - (\overrightarrow{\mathrm{P_1 P_2}} \cdot \overrightarrow{\mathrm{P_1 P_3}})^2} \\
		  & = \frac{1}{2} \sqrt{\|\bm{a}\|^2 \|\bm{b}\|^2 - (\bm{a}\cdot \bm{b})^2 } \\
		  & = \frac{1}{2} \{[(x_2 - x_1)^2+(y_2-y_1)^2+ (z_2 - z_1)^2][(x_3 - x_1)^2+(y_3-y_1)^2+ (z_3 - z_1)^2]\\
		  & \qquad - [(x_2-x_1)(x_3-x_1)+(y_2 -y_1)(y_3-y_1)+(z_2-z_1)(z_3-z_1)]^2\}^{\frac{1}{2}}
	\end{align*}
	である,
\end{leftbar}
	
\newpage

\subsection{p8}

問1.

\begin{leftbar}
	(1) \quad 与えられた直線を$\ell$とする.$\ell$の方程式に$x=-2$を代入すると,$y=2$となるため,$\ell$は点$(-2,2)$を通る.
	また,$\ell$の法線ベクトルのひとつは,
	$
	\left(
		\begin{array}{c}
			2 \\
			3
		\end{array}
	\right)
		$なので,$\ell$の方向ベクトルのひとつは,
		$
		\left(
		\begin{array}{c}
			-3 \\
			2
		\end{array}
		\right)
		$である.よって,$\ell$のベクトル表示のひとつは,
		$
		\left(
		\begin{array}{c}
			x \\
			y
		\end{array}
		\right)
		= \left(
		\begin{array}{c}
		-2 \\
		2
		\end{array}
		\right)
		+t \left(
			\begin{array}{c}
				-3 \\
				2
			\end{array}
			\right)
			~(-\infty < t < \infty)$である.\\

			(2) \quad 与えられた直線を$\ell'$とする.$\ell '$の方向ベクトルのひとつは,$
			\left (
				\begin{array}{c}
					0 \\
					1
				\end{array}
			\right)
			$である.また,$\ell '$は点$(3,0)$を通るので,そのベクトル表示のひとつは,
			$
			\left(
				\begin{array}{c}
					x \\
					y
				\end{array}
			\right)
			= \left(
				\begin{array}{c}
				3 \\
				0
				\end{array}
			\right)
			+t \left(
				\begin{array}{c}
					0 \\
					1
					\end{array}
				\right) ~(-\infty < t < \infty)
				$となる.
		\end{leftbar}

		問2.
		
		\begin{leftbar}
			(1) \par
			与えられたベクトル表示から.
			\begin{align*}
				\begin{cases}
					x=1+2t \\
					y=-1+t
				\end{cases}
			\end{align*}
			であるから,
			\begin{align*}
				\begin{cases}
					t=\frac{x-1}{2} \\
					t=y+1
				\end{cases}
			\end{align*}
			である.これから$t$を消去すると,
			\begin{gather*}
				\frac{x-1}{2} = y+1 \\
				\therefore ~ x-2y-3 =0
			\end{gather*}
			である.\\

			(2) \par 
			点$(-1,-2)$を通り,$x$軸に平行な直線を表すから,$y=-2$が求める直線の方程式である.
		\end{leftbar}
%
\newpage
%
%
%
%
\subsection{p10}
%
問1:
\begin{leftbar}
\begin{align*}
    \begin{cases}
        x+2y+3z=1 \\
        3x+2y+z=-1
    \end{cases}
\end{align*}
から,
\begin{gather*}
    -2x+2z=2 \\
    \therefore \quad -x+z=1
\end{gather*}
である.このとき,$
\left(
    \begin{array}{c}
        x \\
        z
    \end{array}
\right) =
\left(
    \begin{array}{c}
        1 \\
        2
    \end{array}
\right),
\left(
    \begin{array}{c}
        2 \\
        3
    \end{array}
\right)
$はこれを満たす.このときの$y$の値を計算すると,それぞれ$-3$,$-5$なので,結局,与えられた直線は2点$(1,-3,2),(2,-5,3)$を通る.
すなわち,この直線の方向ベクトルのひとつは\[
    \left(
\begin{array}{c}
    2 \\
    -5 \\
    3
\end{array}
    \right)
    -
    \left(
        \begin{array}{c}
            1 \\
            -3 \\
            2
        \end{array}
            \right)
            =
            \left(
                \begin{array}{c}
                    1 \\
                    -2 \\
                    1
                \end{array}
                    \right)
                    \]
である.したがって求めるベクトル表示のひとつは,直線上の任意の位置ベクトルを$\bm{x}$とすると,
\[
    \bm{x}= \left(
        \begin{array}{c}
            1 \\
            -3 \\
            2
        \end{array}
            \right)
            +t
            \left(
            \begin{array}{c}
                1 \\
                -2 \\
                1
            \end{array}
            \right)
            \]
            と表せる.
\end{leftbar}


問2:
\begin{leftbar}
    \begin{proof}
        $t$を$0 \le t \le 1$をみたす実数,線分$\mathrm{P_1 P_2}$上の任意の点の位置ベクトルを$\bm{x}$とする.
        このとき,
        \begin{align*}
            \bm{x} & = \overrightarrow{\mathrm{O P_1}}+t\overrightarrow{\mathrm{P_1 P_2}} \\
            & = \bm{x}_1 + t (\bm{x}_2 - \bm{x}_1) \\
             &= (1-t) \bm{x}_1 + t \bm{x}_2
        \end{align*}
        である.$1-t = t_1$,$t=t_2$と改めておくと,$t$の定め方から$t_1 \ge 0$,$t_2 \ge 0$であり,
        \[
            \bm{x}= t_1 \bm{x}_1 + t_2 \bm{x}_2 ,t_1 + t_2 =1
        \]
        となり,これが証明すべきことであった.
    \end{proof}
\end{leftbar}

\newpage
\subsection{p11~12}

問1:

\begin{leftbar}
    与えられた平面を$(S)$とおく.$(S)$は3点$(-1,0,1)$,$(2,0,-1)$,$(0,-1,0)$を通るので,
    \[
   \bm{x}_1=\left(
        \begin{array}{c}
            -1 \\
            0 \\
            1
        \end{array}
   \right)
   ,\quad \bm{x}_2=\left(
    \begin{array}{c}
        2 \\
        0 \\
        -1
    \end{array}
\right) \quad 
\bm{x}_3=\left(
    \begin{array}{c}
        0 \\
        -1 \\
        0
    \end{array}
\right)
\]
と改めておくと,
\begin{gather*}
    \bm{x}_2 - \bm{x}_1 =
    \left(
        \begin{array}{c}
            3 \\
            0 \\
            -1
        \end{array}
    \right)
    ,\quad 
    \bm{x}_3 - \bm{x}_1 =
    \left(
    \begin{array}{c}
        1 \\
        -1 \\
        -1
    \end{array}
    \right)
\end{gather*}
となり,$\bm{x}_2 - \bm{x}_1$と$\bm{x}_3 - \bm{x}_1$は線型独立なので,求めるベクトル表示のひとつは,
\[
    (S) \colon \bm{x}=\left(
        \begin{array}{c}
            -1 \\
            0 \\
            1
        \end{array}
    \right)+ t 
    \left(
        \begin{array}{c}
            3 \\
            0 \\
            -1
        \end{array}
    \right)+s
    \left(
        \begin{array}{c}
            1 \\
            -1 \\
            -1
        \end{array}
        \right)~( -\infty < t,s<\infty)
        \]
    \end{leftbar}
%
%
%
%
問2:

\begin{leftbar}
    \begin{align*}
        \begin{cases}
            x=1+t-s \\
            y=2-t -2s \\
            z=0+2t+s
        \end{cases}
    \end{align*}
    から$t$と$s$を消去して,
    \[
        x-y-z=-1
    \]
    これが求める直線の方程式である.
\end{leftbar}

問3:

\begin{leftbar}
    \begin{proof}
        \[
            \overrightarrow{\mathrm{OP_1}}=\bm{x}_1,\overrightarrow{\mathrm{OP_2}}=\bm{x}_2,\overrightarrow{\mathrm{OP_3}}=\bm{x}_3
        \]
        とする.このとき,三角形$\mathrm{P_1 P_2 P_3}$上の任意の点の位置ベクトルを$\bm{x}$とすると,
        \begin{gather*}
            \bm{x}=\bm{x}_1 + s(\bm{x}_2 - \bm{x}_1) + t (\bm{x}_3 - \bm{x}_1) \\
            \therefore \quad \bm{x} = (1-s-t)\bm{x}_1 + s\bm{x}_2 + t \bm{x}_3
        \end{gather*}
        となり,$1-s-t=t_1$,$s=t_2$,$t=t_3$と改めて書き直すと,
        \[
            \bm{x} = t_1\bm{x}_1 + t_2\bm{x}_2 + t_3 \bm{x}_3,t_1+t_2+t_3=1
        \]
        である.これが証明すべきことであった.
    \end{proof}
\end{leftbar}
%
\newpage
%
\subsection{p13}
%
問1:
\begin{leftbar}
    $(S_1)$,$(S_2)$の法線ベクトルをそれぞれ$\bm{x}_1$,$\bm{x}_2$とおくと,
    \begin{gather*}
        \bm{x}_1 =
        \left(
            \begin{array}{c}
                1\\
                1\\
                2
            \end{array}
        \right)
        ,
        \bm{x}_2 =
        \left(
            \begin{array}{c}
                3\\
                3\\
                0
            \end{array}
        \right)
        \end{gather*}
        である.ゆえに,交角を$\theta ~(0 \le \theta \le \frac{\pi}{2})$とすると,
        \[
            \cos \theta = \frac{\bm{x}_1 \cdot \bm{x}_2}{\|\bm{x}_1\| \|\bm{x}_2\|}=\frac{3}{3\sqrt{2}}=\frac{1}{\sqrt{2}}
        \]
        であるから,$0 \le \theta \le \frac{\pi}{2}$より$\theta =\frac{\pi}{4}$である.
    \end{leftbar}
%
%
\newpage
%
%
%
\subsection{p18}
%
問: 
\begin{leftbar}
    \begin{proof}
        $A$,$B$,$C$が$2 \times 2$行列の場合を証明する.
        \begin{gather*}
            A=
            \begin{pmatrix}
                a & b \\
                c & d
            \end{pmatrix}
            ,
            B=
            \begin{pmatrix}
                e & f \\
                g & h
            \end{pmatrix}
            ,C=
            \begin{pmatrix}
                i & j \\
                k & l
            \end{pmatrix}
        \end{gather*}
        とし,$A$,$B$,$C$の成分はすべて複素数であるとする.このとき,
        \begin{align*}
            (AB)C & =
            \begin{pmatrix}
                ae+bg & af+bh \\
                ce+dg & cf+dh
            \end{pmatrix}
            \begin{pmatrix}
                i & j \\
                k & l
            \end{pmatrix}
            \\
            & =
            \begin{pmatrix}
                aei +bgi +afk +bhk & aej+bgj+afl+bhl \\
                cei +dgi+cfk +dhk & cej +dgj + cfl +dhl
            \end{pmatrix}
        \end{align*}
            となる.他方
            \begin{align*}
                A(BC)& =
                \begin{pmatrix}
                    a & b \\
                    c & d
                \end{pmatrix}
                \begin{pmatrix}
                    ei + fk & ej +fl \\
                    gi + hk & gj +hl 
                \end{pmatrix}
                \\
                & =
                \begin{pmatrix}
                    aei + afk +bgi +bhk & aej +afl +bgj +bhl \\
                    cei + cfk + dgi +dhk & cej + cfl + dgi +dhl
                \end{pmatrix}
            \end{align*}
            となり,たしかに$(AB)C=A(BC)$である.
        \end{proof}
    \end{leftbar}
%
\newpage
\subsection{p19}
%
問1:
%
\begin{leftbar}
	\begin{proof}
		\[
			\begin{pmatrix}
				-1 & 0\\
				0 & -1
			\end{pmatrix}
			\left(
				\begin{array}{c}
					x \\
					y 
				\end{array}
			\right)
			=
			\left(
				\begin{array}{c}
					-x \\
					-y 
				\end{array}
			\right)
			\]
			となり,これは明らかに線型変換である.対応する行列は,$
			\begin{pmatrix}
				-1 & 0 \\
				0 & -1
			\end{pmatrix}
			$である.
		\end{proof}
	\end{leftbar}
%
	問2:
	\begin{leftbar}
		\begin{proof}
			式(15)より,$2 \times 2$行列$A$,$B$とベクトル$\bm{x}$について,
			\begin{align*}
				T_B (T_A (\bm{x})) &= B(A\bm{x}) \\
				& = (BA) \bm{x} \\
				& = T_{BA} (\bm{x})
			\end{align*}
			である.これが証明すべきことであった.
		\end{proof}
    \end{leftbar}
    \newpage
\begin{leftbar}
$\bm{x} =
\left(
\begin{array}{c}
x \\
y
\end{array}
\right)
$
とおくと,(17)式より,
\begin{align*}
T \bm{x} &= \frac{ax+by}{a^2+b^2} \bm{a} \\
&=\left(
\begin{array}{c}
a^2x +aby \\
ab x + b^2y \\
\end{array}
\right)
\\
&=
\begin{pmatrix}
a^2 & ab \\
ab & b^2
\end{pmatrix}
\left(
\begin{array}{c}
x \\
y
\end{array}
\right) \\
&= 
\begin{pmatrix}
a^2 & ab \\
ab & b^2
\end{pmatrix}
\bm{x}
\end{align*}
であるから,
\[
T=\begin{pmatrix}
a^2 & ab \\
ab & b^2
\end{pmatrix}
\]
となる.
\end{leftbar}
%
\newpage
%
%
%
\subsection{p22〜23}
%
%
%
\newpage
%
%
%
\setcounter{equation}{0}
\subsection{p29〜30(章末問題)}
%
%
%
問2\\
\noindent
\begin{leftbar}
\begin{proof}
実際に行列式を計算してみると,
\begin{eqnarray}
\left |
\begin{array}{ccc}
1 & 1 & 1 \\
x & x_1 & x_2 \\
y & y_1 & y_2 
\end{array}
\right |
& = & 1 \cdot x_1 \cdot y_2 +1 \cdot x_2 \cdot y + 1 \cdot x \cdot y_1 - 1 \cdot x_2 \cdot y_1 - 1 \cdot x \cdot y_2 - 1 \cdot x_1 \cdot y \\
\label{1}
& = & y(x_2 -x_1) + x ( y_1 - y_2 ) +x_1 y_2 - x_2 y_1
\end{eqnarray}
\eqref{1} について,$x_1 \neq x_2$のとき,
\begin{eqnarray}
& & y(x_2 -x_1) + x ( y_1 - y_2 ) +x_1 y_2 - x_2 y_1=0 \\
\label{2}
& \therefore &y=  \left( \frac{y_2 - y_1}{x_2 -x_1} \right)x  + \frac{x_2 y_1 - x_1 y_2 }{x_2 -x_1}
\end{eqnarray}
\eqref{2}に$x=x_1$を代入すると,
\begin{eqnarray}
y&=& \frac{x_1 y_2 -y_1 x_1 +x_2 y_1 - x_1 y_2}{x_2-x_1} \\
& = & \frac{(x_2 - x_1)y_1}{x_2 - x_1} \\
&=&y_1
\end{eqnarray}
\eqref{2}に$x=x_2$を代入すると,同様に$y=y_2$が得られる.ゆえに,たしかに直線を表す.\\
$x_1 = x_2$のとき,\eqref{1}式は,
\begin{equation}
x(y_1 - y_2) = -x_1 y_2 + x_2 y_1
\end{equation}
となり,$P_1$,$P_2$が相異なることより$y_1 \neq y_2$なので,$x$はつねに定数となり,直線となる.\\
以上により,問題の主張が正しいことが示された.
\end{proof}
\end{leftbar}

\newpage
%
\subsection{p30}
%
問10:\\
 イ)
\begin{leftbar}
    \begin{proof}
    単位ベクトル$\bm{e}_1$,$\bm{e}_2$,$\bm{e}_3$を適当にとり,
    \[
    \bm{a} = \alpha_1 \bm{e}_1,~\bm{b} = \beta_1 \bm{e}_1+\beta_2 \bm{e}_2,~\bm{c}= \gamma_1 \bm{e}_1 + \gamma_2 \bm{e}_2 + \gamma_3 \bm{e}_3
    \]
    とおく.このとき,
        \begin{align*}
            (\bm{a} \times \bm{b}) \times \bm{c} &= \alpha_1 \beta_2 \bm{e}_3 \times (\gamma_1 \bm{e}_1 + \gamma_2 \bm{e}_2 + \gamma_3 \bm{e}_3) \\
            & = \alpha_1 \beta_2 \gamma_1 \bm{e}_2 - \alpha_1 \beta_2 \gamma_2 \bm{e}_1 \\
            & = -(\bm{b},\bm{c})\bm{a}+(\bm{a},\bm{c}) \bm{b}
        \end{align*}
        であり,これが証明すべきことであった.
    \end{proof}
    \end{leftbar}
\newpage
%
\section{第2章:行列(解答)}
\subsection{p34}
問1\\
\noindent
$\textsl{Hint}:単純な行列の和・積に関する命題は,両辺の(i,j)成分を比較する方法が1つある.今回はそれを使う命題であるが,(i,j)成分を計算する際,左辺は積の形になっているので,どうしても\sum を使わざるを得ない.その時は公式に当てはめるなど頭を使わない方法ではなく,丁寧に成分ごとで1つ1つ計算していき,最後に\sum でまとめて証明として書き出せばいい.$
\begin{leftbar}
\begin{proof}
~\\
後半二つの主張は明らか.また,二つ目の主張は一つ目の主張と同様にして示すことができるので,一つ目のみ示すことにする.\\
$A=(a_{pq})をk \times l行列,B= (b_{qr}),C=(c_{qr})をl \times m$行列とする.示したい式の両辺がともに定義され,ともに$k \times m行列であることはよい.行列B+Cの(q,r)成分はb_{qr}+c_{qr}であるから,左辺の(p,r)成分は,$
\begin{eqnarray*}
\sum_{q=1}^{l}a_{pq}\left(b_{qr}+c_{qr}\right)=\sum_{q=1}^{l}a_{pq}b_{qr}+\sum_{q=1}^{l}a_{pq}c_{qr}
\end{eqnarray*}
と書ける.この等号の右辺は$ABの(p,r)成分とACの(p,r)成分の和である.$これより,主張が示された.
\end{proof}
\end{leftbar}
%
%
%
\newpage
%
%
%
\subsection{p41}
%
%
%
\newpage
%
%
%
\subsection{p42}
%
%
%
\newpage
%
%
%
\subsection{p48}
%
%
%
\newpage
%
%
%
\subsection{p53}
%
%
%
\newpage
%
%
%
\subsection{p62〜63}
%
%
%
\newpage
%
%
%
\subsection{p65}
%
%
%
\newpage
%
%
%
\subsection{p70(章末問題)}
問6\\
\noindent
\begin{leftbar}
\begin{proof}
~\\
イ)$A_k=Eなるkが存在したとする.k = 1 ならば明らか.$\\
$k>2ならば,A^k=A·A^{k-1}=A^{k-1}· A = E となるので,A^{k-1}がAの逆行列となるので,Aは正則.$\\
\\
$ロ)Aが正則であるとする.A^2=Aの両辺にA^{-1}をかけると,A=Eとなるが,これは仮定に反する.$\\
\\
$ハ)A^k=Oとなる k が存在したとして,A が正則であるとする.この等式の両辺に (A^{−1})^kをかけるとE=Oとなるが,これは明らかに矛盾.$\\
\\
$ニ)E−A に関してのみ示す.他方も同様にして示せる.A^{k}=Oとなるkが存在したとする.$\\
$すると$
\begin{eqnarray*}
(E − A)(E+A+ \cdots +A^{k−1})=(E+A+ \cdots + A^{k−1})(E−A)=E−A^k=E
\end{eqnarray*}
$となることから主張が従う.$
\end{proof}
\end{leftbar}
%
\newpage
%
%
%
\setcounter{equation}{0}
\subsection{p71}
%
%
%

問1:\\
\noindent
\begin{leftbar}
\begin{proof}
$w,x,y,z \in \mathbb{C}$として,$X=
\begin{pmatrix}
w & x \\
y & z
\end{pmatrix}
$とおく.このとき,
\begin{align}
AX & =
\begin{pmatrix}
1 & 2 \\
2 & 4
\end{pmatrix}
\begin{pmatrix}
w & x \\
y & z
\end{pmatrix}
\\
& = 
\begin{pmatrix}
w + 2y & x + 2z \\
2w + 4y & 2x + 4z
\end{pmatrix}
\end{align}
となり,
\begin{gather*}
\begin{pmatrix}
w + 2y & x +2z \\
2w + 4y & 2x + 4z
\end{pmatrix}
=
\begin{pmatrix}
1 & 0 \\
0 & 1 
\end{pmatrix}
\end{gather*}
から,
\begin{gather*}
\begin{cases}
w+2y = 1 \\
x+2z = 0 \\
2w+4y =0 \\
2x+4z=1
\end{cases}
\end{gather*}
簡単な計算により,この連立方程式を満たす$w,x,y,z \in \mathbb{C}$は存在しない.これで前半の主張が示された.\par
後半について,$h,i,j,k \in \mathbb{C}$として,$Y=
\begin{pmatrix}
h & i \\
j & k
\end{pmatrix}
$
とすると,
\begin{align}
YA=&
\begin{pmatrix}
h & i \\
j & k
\end{pmatrix}
\begin{pmatrix}
1 & 2 \\
2 & 4
\end{pmatrix}
\\
 =&
\begin{pmatrix}
h+2 i & 2h + 4i \\
j + 2k & 2j +4k
\end{pmatrix}
\end{align}
となる.
\begin{gather*}
\begin{pmatrix}
h+2 i & 2h + 4i \\
j + 2k & 2j +4k
\end{pmatrix}
=
\begin{pmatrix}
1 & 0 \\
0 & 1 
\end{pmatrix}
\end{gather*}
から,
\begin{gather*}
\begin{cases}
h+2 i =1 \\
2h + 4i =0 \\
j + 2k =0 \\
2j +4k =1
\end{cases}
\end{gather*}
簡単な計算により,この連立方程式を満たす$h,i,j,k \in \mathbb{C}$は存在しない.これで後半の主張も示された.
\end{proof}
\end{leftbar}
%
\newpage
\setcounter{equation}{0}
問7:
\begin{leftbar}
\begin{proof}
$X=(x_{ij})$,$Y=(y_{ij})$とする.
ここで,$XY$の$(i,i)$成分は$\sum_{j=1}^{n} x_{ij} y_{ji}$であるから,
\begin{equation}
\label{問7:1}
\mathrm{Tr}(XY)  =  \sum_{i=1}^{n} \left( \sum_{j=1}^{n} x_{ij} y_{ji} \right)
\end{equation}
となる.他方,$YX$については,同様の議論により,
\begin{eqnarray}
\mathrm{Tr} (YX) & = & \sum_{i=1}^{n} \left( \sum_{j=1}^{n}  y_{ij} x_{ji} \right) \nonumber \\
\label{問7:2}
 & = & \sum_{i=1}^{n} \left( \sum_{j=1}^{n}  x_{ji} y_{ij}  \right)
\end{eqnarray}
\eqref{問7:2}式で$i$と$j$をおきかえれば,
\begin{equation}
\label{問7:3}
\mathrm{Tr} (YX) = \sum_{j=1}^{n} \left( \sum_{i=1}^{n}  x_{ij} y_{ji}  \right)
\end{equation}
\eqref{問7:1}式と\eqref{問7:3}式を比較して,
\begin{equation}
\label{問7:4}
\mathrm{Tr} (XY) = \mathrm{Tr} (YX)
\end{equation}
を得て,\eqref{問7:4}式より$\mathrm{Tr} (XY-YX) =0$であるが,一方,$\mathrm{Tr} (E_n) =n \neq 0$であるため,これは矛盾.\\
ゆえに,$XY-YX=E_n$となる$n$次行列$X$,$Y$は存在しないことが示された.
\end{proof}
\end{leftbar}

\newpage

問11\\
\noindent
\begin{proof}
\begin{leftbar}
~\\
$Pをn次正方行列で$
\begin{eqnarray*}
^t PP=E\\
P+Eは正則
\end{eqnarray*}
$を満たしているとする.$\\
1.$(P − E)(P + E) = (P + E)(P − E)右辺と左辺をそれぞれ計算する.$
\begin{eqnarray*}
& (右辺) = P^2+P−P−E=P^2−E\\
& (右辺) = P^2−P+P−E=P^2−E
\end{eqnarray*}
2.$A := (P − E)(P + E)^{−1}とおくと,^tA=−A$
\begin{eqnarray*}
^tA&=&^t((P−E)(P+E)^{−1})\\
&=&^t((P+E)^{-1})^t(P−E)\\
&=&(^t(P+E))^{-1}(^tP−E)\\
&=&(^tP+^tPP)^{-1}(^tP−^tPP)\\
&=&(^tP(E + P))^{-1}(^tP(E − P))\\
&=&(E + P)^{−1}(^tP)^{−1}(^tP)(E − P)\\
&=&(E + P)^{−1}(E − P)\\
&=&−(P + E)^{−1}(P − E)\\
\end{eqnarray*}
$仮定P+Eの正則性より,(1)の両辺にP+Eの逆行列を左, 右から掛ければ$
\begin{eqnarray*}
& (P + E)^{-1}(P − E)(P + E)(P + E)^{−1}=(P + E)^{−1}(P + E)(P − E)(P + E)^{−1}\\
& (P + E)^{−1}(P − E)=(P − E)(P + E)^{−1}
\end{eqnarray*}
が成り立つことから,
\begin{eqnarray*}
^tA = −(P + E)^{−1}=(P − E) = −(P − E)(P + E)^{−1}= −A
\end{eqnarray*}
となる.\\
\\
3.$E − A は正則$\\
$\dfrac{1}{2}(P + E) が逆行列であることを示す.$
\begin{eqnarray*}
(E−A)(P+E)&=&P+E−A(P+E)\\
&=&P+E+(−(P − E)(P + E)^{−1})(P + E)\\
&=&P+E−P+E \\
&=&2E
\end{eqnarray*}
より, 題意は示される.
\end{leftbar}
%
%
%
\newpage
%
%
%
\begin{leftbar}
4.$P= (E + A)(E − A)^{−1}$
\begin{eqnarray*}
(E + A)(E − A)^{−1}&=&(E − P + P + A)\frac{1}{2}(P + E)\\
&=&\frac{1}{2}(E − P)(E + P)+\frac{1}{2}P(P + E)+\frac{1}{2}A(P + E)\\
&=&\frac{1}{2}(E − P^2) +\frac{1}{2}(P^2+P)+\frac{1}{2}(P − E)(P + E)^{−1}(P + E)\\
&=&\frac{1}{2}(E − P^2)+\frac{1}{2}(P^2+P)+\frac{1}{2}(P − E)\\
&=&P
\end{eqnarray*}
\end{leftbar}
\end{proof}
以上より題意は示された.
%
%
%
\newpage
%
%
%
\section{第3章:行列式(解答)}
\subsection{p83}
%
%
%
\newpage
%
%
%
\subsection{p90(章末問題)}
問10\\
\noindent
\begin{leftbar}
\begin{proof}
必要性・十分性をそれぞれ証明する.\\ \\
1.$Aが正則かつA^{−1}が整数行列であると仮定し,\det A=\pm 1であることを示す.$\\
$Aは整数行列であり,その行列式は,各要素の和と積でかけているから\det A \in \mathbb{Z}である.同様にして\det (A^{−1}) \in \mathbb{Z}である.逆行列の行列式は,$
\begin{eqnarray*}
\det (A^{−1})=\frac{1}{\det A}
\end{eqnarray*}
$が成り立つ..つまり,\det A,1/\det A \in \mathbb{Z} である.これを満たす整数は\pm 1だけである.$\\
\\
2.$\det A=\pm 1であることを仮定し,Aが正則かつA^{−1}が整数行列であることを示す.$\\
$\det A \neq 0 より A の正則性がわかる.また,Aの余因子行列を\tilde{A}とすると,余因子はAの各要素の和と積によって表現される.つまり,余因子は整数であるから\tilde{A}は整数行列である. また$
\begin{eqnarray*}
A^{−1}=\frac{1}{\det A}\tilde{A}
\end{eqnarray*}
$となる.\det A=\pm 1であり, 余因子は整数であるから,A^{−1}は整数行列である.$
\end{proof}
\end{leftbar}
%
%
%
\newpage
%
%
%
\section{第4章:線型空間(解答)}
\subsection{p94}
問.\\
\noindent
\textsl{Hint}:集合$A$の元$a,b,c$に対して,同値関係であることはP93の定義から反射律・対称律・推移律が成立することなので,これを示せばよい.商集合の元の個数は商集合の定義に振り返る.\\ \\
まず,同値関係であることを示す.$A,B \in \bm{M_{m,n}},P,Q,R,Sを正則行列とする.$\\
基本行列の積は正則行列で表されることを用いて\\
・$E_{m}AE_{n}=A~~(反射律)$ \\
・$PAQ=Bならば,A=P^{-1}BQ^{-1}~~(対称律)$\\
・$PAQ=B,RBS=Cならば,PAQ=R^{-1}CS^{-1} \Longleftrightarrow (PR)A(QS)=C$~~(推移律)\\
より同値関係である.\\
また,同値関係ごとに類別した類は正則行列を掛けることにより,ある標準形$F_{m,n}(r)$に直すことができる$(m,n)$型行列全体である.これを$C(r)$とおく.よって商集合はこの類全体の集合なので,$C(0),C(1),\cdots ,C({\rm min}(m,n))$である.よって元の個数は$1+{\rm min}(m,n)$個である.
%
%
%
\newpage
%
%
%
\subsection{p106}
\noindent
問1:\\
\textsl{Hint}:単なる計算問題なので手順通り求めればよいが,求めるときはp106の(3)式が有用である.\\ \\
求める$E→F$の取り替え行列を$P=(p_{ij})$とし,
\begin{eqnarray*}
\bm{e}_1=
\begin{pmatrix}
1 \\
0 \\
1
\end{pmatrix}
,
\bm{e}_2=
\begin{pmatrix}
2 \\
1 \\
0
\end{pmatrix}
,
\bm{e}_3=
\begin{pmatrix}
1 \\
1 \\
1
\end{pmatrix}
,
\bm{f}_1=
\begin{pmatrix}
3 \\
-1 \\
4
\end{pmatrix}
,
\bm{f}_2=
\begin{pmatrix}
4 \\
1 \\
8
\end{pmatrix}
,
\bm{f}_3=
\begin{pmatrix}
3 \\
-2 \\
6
\end{pmatrix}
\end{eqnarray*}
とする.~~p106の(3)式から
\begin{eqnarray*}
\bm{f}_i=\sum^{3}_{j=1}p_{ji}\bm{e}_{j}=p_{1i}\bm{e}_1+p_{2i}\bm{e}_2+p_{3i}\bm{e}_3
\end{eqnarray*}
それぞれ$i=1,2,3$について連立方程式を作ると
\begin{eqnarray*}
f_1=p_{11}\bm{e}_1+p_{21}\bm{e}_2+p_{31}\bm{e}_3 \\
f_2=p_{12}\bm{e}_1+p_{22}\bm{e}_2+p_{32}\bm{e}_3 \\
f_3=p_{13}\bm{e}_1+p_{23}\bm{e}_2+p_{33}\bm{e}_3
\end{eqnarray*}
これを解くことにより$p_{11}=\frac{9}{2},p_{21}=-\frac{1}{2},p_{31}=-\frac{1}{2},p_{12}=5,p_{22}=-2,p_{32}=3,p_{13}=\frac{13}{2},p_{23}=-\frac{3}{2},p_{33}=-\frac{1}{2}$なので
\begin{eqnarray*}
P=
\begin{pmatrix}
9/2 & 5  & 13/2 \\
-1/2 & -2 & -3/2 \\
-1/2 & 3 & -1/2
\end{pmatrix}
\end{eqnarray*}
である.またp106の(3)式を簡単な計算をすることにより
\begin{eqnarray*}
(\bm{f}_1,\bm{f}_2,\cdots ,\bm{f}_n)=(\bm{e}_1,\bm{e}_2,\cdots ,\bm{e}_n)P
\end{eqnarray*}
であるから
\begin{eqnarray*}
\begin{pmatrix}
3 & 4 & 3 \\
-1 & 1 & -2 \\
4 & 8 & 6 \\
\end{pmatrix}
=
\begin{pmatrix}
1 & 2 & 1 \\
0 & 1 & 1 \\
1 & 0 & 1 \\
\end{pmatrix}
P \\ 
%
%
%
P=
\begin{pmatrix}
3 & 4 & 3 \\
-1 & 1 & -2 \\
4 & 8 & 6 \\
\end{pmatrix}
\begin{pmatrix}
1 & 2 & 1 \\
0 & 1 & 1 \\
1 & 0 & 1 \\
\end{pmatrix}
^{-1}
\end{eqnarray*}
から求めることもできる.\\ \\

\newpage

問2: \\
\textsl{Hint}:問1と同様に計算すればよい.本解答では問1の最後で紹介した計算方法では計算できないことに注意したい.\\ \\
p106の(3)式から
\begin{eqnarray*}
\bm{f}_i=\sum^{2}_{j=1}p_{ji}\bm{e}_{j}=p_{1i}\bm{e}_1+p_{2i}\bm{e}_2
\end{eqnarray*}
よって
\begin{eqnarray*}
& \bm{f}_1=p_{11}\bm{e}_1+p_{21}\bm{e}_2 \\
& \bm{f}_2=p_{12}\bm{e}_1+p_{22}\bm{e}_2 \\ \\
%%%
&
\begin{pmatrix}
0 \\
1 \\
-1 \\
\end{pmatrix}
=
p_{11}
\begin{pmatrix}
1 \\
-1 \\
0 \\
\end{pmatrix}
+p_{21}
\begin{pmatrix}
1 \\
0 \\
-1 \\
\end{pmatrix}
\\ \\
&
\begin{pmatrix}
1 \\
1 \\
-2 \\
\end{pmatrix}
=
p_{12}
\begin{pmatrix}
1 \\
-1 \\
0 \\
\end{pmatrix}
+p_{22}
\begin{pmatrix}
1 \\
0 \\
-1 \\
\end{pmatrix}
\end{eqnarray*}
よって$p_{11}=-1,p_{21}=1,p_{12}=-1,p_{22}=2$であるから基底の取り替え$E→F$の行列は
\begin{eqnarray*}
P=
\begin{pmatrix}
-1 & -1 \\
1 & 2 \\
\end{pmatrix}
\end{eqnarray*}
である.
%
%
%
\newpage
%
%
%
\subsection{p107-108}
\noindent
問1:\\
\textsl{Hint}:部分空間の定義P107の(1)式から部分空間の判定を行う.次元は単に基底の個数を数えればよい.部分空間でないときは単に反例を示せばいい.\\
%%%
\dotfill \\
%%%
イ)この$\bm{x}全体は$部分空間をなす.実際,この条件を満たす空間を$Wとし,\bm{x,y} \in W$とする.
\begin{eqnarray*}
\bm{x}+\bm{y}=
\begin{pmatrix}
x_1+y_1 \\
x_2+y_2 \\
\vdots \\
x_n+y_n
\end{pmatrix}
,~~~
a\bm{x}=
\begin{pmatrix}
ax_1 \\
ax_2 \\
\vdots \\
ax_n
\end{pmatrix}
\end{eqnarray*}
条件から$(x_1+y_1)+(x_2+y_2)+\cdots (x_n+y_n)=(x_1+x_2+\cdots +x_n)+(y_1+y_2+\cdots +y_n)=0$\\
$ax_1+ax_2+ \cdots +ax_n=a(x_1+x_2+\cdots x_n)=0であるから,\bm{x}+\bm{y} \in W,~a\bm{x} \in W$である.よって,この空間は部分空間をなす.また次元は
\begin{eqnarray*}
\bm{x}=
\begin{pmatrix}
x_1 \\
x_2 \\
\vdots \\
x_{n-1} \\
x_n
\end{pmatrix}
=
\begin{pmatrix}
x_1 \\
x_2 \\
\vdots \\
x_{n-1} \\
-(x_1+x_2+\cdots x_{n-1})
\end{pmatrix}
=
x_1
\begin{pmatrix}
1 \\
0 \\
\vdots \\
0 \\
-1
\end{pmatrix}
+x_2
\begin{pmatrix}
0 \\
1 \\
\vdots \\
0 \\
-1
\end{pmatrix}
+\cdots +x_{n-1}
\begin{pmatrix}
0 \\
0 \\
\vdots \\
1 \\
-1\\
\end{pmatrix}
\end{eqnarray*}
により,明らかに$n-1$である.\\
%%%
\dotfill \\
%%%
ロ)この$\bm{x}全体は部分空間をなす.$実際,この条件を満たす空間を$Wとし,\bm{x,y} \in W$とする.
\begin{eqnarray*}
\bm{x}+\bm{y}=
\begin{pmatrix}
x_1+y_1 \\
x_2+y_2 \\
\vdots \\
x_{p}+y_{p} \\
x_{p+1}+y_{p+1} \\
\vdots \\
x_n+y_n
\end{pmatrix}
,~~~
a\bm{x}=
\begin{pmatrix}
ax_1 \\
ax_2 \\
\vdots \\
ax_{p} \\
ax_{p+1} \\
\vdots \\
ax_n
\end{pmatrix}
\end{eqnarray*}
条件から$x_{p+1}+y_{p+1}=x_{p+2}+y_{p+2}=\cdots =x_{n}+y_{n}=0,ax_{p+1}=ax_{p+2}=\cdots =ax_n=0である.$\\
よって$\bm{x}+\bm{y} \in W, a\bm{x} \in W$であるから,この空間は部分空間をなす.また次元は
\begin{eqnarray*}
\bm{x}=
\begin{pmatrix}
x_1 \\
x_2 \\
\vdots \\
x_{p} \\
x_{p+1} \\
\vdots \\
x_n \\
\end{pmatrix}
=
\begin{pmatrix}
x_1 \\
x_2 \\
\vdots \\
x_{p} \\
0 \\
\vdots \\
0 \\
\end{pmatrix}
=x_1
\begin{pmatrix}
1 \\
0 \\
\vdots \\
0 \\
0 \\
\vdots \\
0 \\
\end{pmatrix}
+x_2
\begin{pmatrix}
0 \\
1 \\
\vdots \\
0 \\
0 \\
\vdots \\
0 \\
\end{pmatrix}
+\cdots +x_p
\begin{pmatrix}
0 \\
0 \\
\vdots \\
1 \\
0 \\
\vdots \\
0 \\
\end{pmatrix}
\end{eqnarray*}
により$p$である.\\
%%%
\dotfill \\
%%%
ハ)この$\bm{x}全体$は部分空間をなさない.実際
\begin{eqnarray*}
\bm{x}=
\begin{pmatrix}
1 \\
0 \\
0 \\
\vdots \\
0
\end{pmatrix}
,~~~~
\bm{y}=
\begin{pmatrix}
0 \\
1 \\
0 \\
\vdots \\
0
\end{pmatrix}
\end{eqnarray*}
とすると
\begin{eqnarray*}
\bm{x}+\bm{y}=
\begin{pmatrix}
1 \\
1 \\
0 \\
\vdots \\
0
\end{pmatrix}
\end{eqnarray*}
条件に当てはめると$1^2+1^2=2 \neq 1$より,加法に関して(また示していないがスカラー倍に関しても)閉じていないので$\bm{x}+\bm{y} \not\in W$である.\\
%%%
\dotfill \\
%%%
ニ)この$\bm{x}全体$は部分空間をなす.$実際,この条件を満たす空間をW,c \in \bm{K}とし,\bm{x,y} \in W$とする.\\
内積の定義から
\begin{eqnarray*}
(\bm{a},\bm{x}+\bm{y})=(\bm{a},\bm{x})+(\bm{a},\bm{y})=0 ,~~~~~(\bm{a},c\bm{x})=c(\bm{a},\bm{x})=0
\end{eqnarray*}
よって$\bm{x}+\bm{y} \in W, c\bm{x} \in W$から部分空間をなす.次元を求めるため,$\bm{a}=(a_{i})$とおく.
\begin{itemize}
\item $\bm{a} \neq \bm{o}のとき$
\begin{eqnarray*}
(\bm{a},\bm{x})=a_1x_1+a_2x_2+ \cdots +a_nx_n=0
\end{eqnarray*}
より$a_n \neq 0$のとき
\begin{eqnarray*}
x_n=-\frac{a_1}{a_n}x_{1}-\frac{a_2}{a_n}x_{2}- \cdots -\frac{a_{n-1}}{a_n}x_{n-1}
\end{eqnarray*}
となり,このときの部分空間$Wの次元はn-1$である.(明らかなので,詳細は略)\\
また$a_n=0$のとき
\begin{eqnarray*}
(\bm{a},\bm{x})=a_1x_1+a_2x_2+ \cdots +a_{n-1}x_{n-1}=0
\end{eqnarray*}
と書き直せ,$a_{n-1} \neq 0$のとき
\begin{eqnarray*}
x_{n-1}=-\frac{a_1}{a_{n-1}}x_{1}-\frac{a_2}{a_{n-1}}x_{2}- \cdots -\frac{a_{n-2}}{a_{n-1}}x_{n-2}
\end{eqnarray*}
となる.このときの部分空間$Wの次元はn-1である.$よって帰納的に$\bm{a}$に0でない成分が存在するとき,次元は$n-1$である.\\
\item $\bm{a} \neq \bm{o}のとき,(\bm{a},\bm{x})は\bm{x}の成分に依らず0となる.よって次元はnである.$
\end{itemize}
よって次元は
\begin{eqnarray*}
{\rm dim}~W=
\left\{
\begin{array}{l}
n-1~~(\bm{a} \neq \bm{o}) \\
~~~~~n~~(\bm{a}=\bm{o})
\end{array}
\right.
\end{eqnarray*}
となる.\\
%%%
\dotfill \\
\newpage
\noindent
%
問2:\\
\textsl{Hint}:直感で部分空間とならない集合が分からない場合は,具体例を何個か上げて反例が無いか調べる.大丈夫そうなら実際に証明してみる方向でいいだろう.\\
%%%
\dotfill \\
%%%
イ)これは部分空間をなさない.実際,非正則行列全体の集合を$Wとし,A,B \in W$を
$
A=
\begin{pmatrix}
1 & 0 \\
0 & 0 \\
\end{pmatrix}
,~~
B=
\begin{pmatrix}
0 & 0 \\
0 & 1 \\
\end{pmatrix}
$
とすると
\begin{eqnarray*}
A+B=
\begin{pmatrix}
1 & 0 \\
0 & 1 \\
\end{pmatrix}
\end{eqnarray*}
これは正則行列であるから,$A+B \not\in W$である.よって加法に関して閉じていないので部分空間をなさない.\\
%%%
\dotfill \\
%%%
ロ)これは部分空間をなす.実際,$AX=XBとなるX全体の集合をWとし,X,Y \in W$とすると
\begin{eqnarray*}
AX=XB \\
AY=YB
\end{eqnarray*}
この2式の辺々を足したり,第1式の両辺を$a$倍することで
\begin{eqnarray*}
A(X+Y)=(X+Y)B \\
A(aX)=(aX)B
\end{eqnarray*}
$X+Y,~aXを1つの行列として見ればX+Y \in W,~aX \in Wである.よって部分空間をなす.$\\
%%%
\dotfill \\
%%%
ハ)これは部分空間をなさない.実際,冪零行列全体の集合を$Wとし,A,B \in Wを$
$
A=
\begin{pmatrix}
1 & -1 \\
1 & -1 \\
\end{pmatrix}
,~~
B=
\begin{pmatrix}
1 & 1 \\
-1 & -1 \\
\end{pmatrix}
$
とおくと
\begin{eqnarray*}
A+B=
\begin{pmatrix}
2 & 0 \\
0 & -2 \\
\end{pmatrix}
\end{eqnarray*}
これは冪零行列ではないので,$A+B \not\in Wである.よって加法に関して閉じてないので部分空間をなさない.$\\
%%%
\dotfill \\
%%%
ニ)これは部分空間をなさない.実際,整数を成分とする行列全体を$Wとし,$
$
A=
\begin{pmatrix}
1 & 1 \\
1 & 1 \\
\end{pmatrix}
\in W
$
とする.
\begin{eqnarray*}
\frac{1}{2}A=
\begin{pmatrix}
1/2 & 1/2 \\
1/2 & 1/2 \\
\end{pmatrix}
\not\in W
\end{eqnarray*}
よりスカラー倍に関して閉じていないので部分空間をなさない.\\
%%%
\dotfill \\
\newpage
%%%
\subsection{p121}
問:\\
\textsl{Hint}:p127にある定理をそのまま使って,どの2つのベクトルも直交するようなベクトルを作り,大きさを1にするように計算する.具体的に解き方は以下の解法を参照.\\
%%%
\dotfill \\ \\
%%%
$\bm{a}_1=
\begin{pmatrix}
1 \\
-1 \\
0
\end{pmatrix}
,\bm{a}_2=
\begin{pmatrix}
1 \\
0 \\
-1
\end{pmatrix}
,\bm{a}_3=
\begin{pmatrix}
1 \\
2 \\
3
\end{pmatrix}
とおいて,シュミットの直交化法によって作られる,どの2つのベクトルも直交するようなベクトルを\bm{b}_1, \bm{b}_2, \bm{b}_3 ,正規直交基底を\bm{c}_1, \bm{c}_2, \bm{c}_3 とする.$
\begin{eqnarray*}
& \bm{b}_1=\bm{a}_1=
\begin{pmatrix}
1 \\
-1 \\
0 \\
\end{pmatrix}
\longrightarrow
\bm{c}_1=\frac{1}{\sqrt{2}}
\begin{pmatrix}
1 \\
-1 \\
0 \\
\end{pmatrix}
\\ \\
& \bm{b_2}=\bm{a}_2-(\bm{a}_2, \bm{c}_1)\bm{c}_1=
\begin{pmatrix}
1 \\
0 \\
-1 \\
\end{pmatrix}
-\frac{1}{2}
\begin{pmatrix}
1 \\
-1 \\
0 \\
\end{pmatrix}
=\frac{1}{2}
\begin{pmatrix}
1 \\
1 \\
-2 \\
\end{pmatrix}
\longrightarrow
\bm{c}_2 =\frac{1}{\sqrt{6}}
\begin{pmatrix}
1 \\
1 \\
-2 \\
\end{pmatrix}
\\ \\
& \bm{b}_3=\bm{a}_3-(\bm{a}_3, \bm{c}_1)\bm{c}_1 -(\bm{a}_3, \bm{c}_2)\bm{c}_2=
\begin{pmatrix}
1 \\
2 \\
3 \\
\end{pmatrix}
+\frac{1}{2}
\begin{pmatrix}
1 \\
-1 \\
0 \\
\end{pmatrix}
+\frac{1}{2}
\begin{pmatrix}
1 \\
1 \\
-2 \\
\end{pmatrix}
=
\begin{pmatrix}
2 \\
2 \\
2 \\
\end{pmatrix}
\longrightarrow 
\bm{c}_3 =\frac{1}{\sqrt{3}}
\begin{pmatrix}
1 \\
1 \\
1 \\
\end{pmatrix}
\end{eqnarray*}
よって正規直交基底は
\begin{eqnarray*}
\left\{
\bm{c}_1=\frac{1}{\sqrt{2}}
\begin{pmatrix}
1 \\
-1 \\
0 \\
\end{pmatrix}
,
\bm{c}_2 =\frac{1}{\sqrt{6}}
\begin{pmatrix}
1 \\
1 \\
-2 \\
\end{pmatrix}
,
\bm{c}_3 =\frac{1}{\sqrt{3}}
\begin{pmatrix}
1 \\
1 \\
1 \\
\end{pmatrix}
\right\}
\end{eqnarray*}
\dotfill \\
【参考】正規直交基底を求めるシュミットの直交化法は次のように書き直せる.
\begin{eqnarray*}
& \bm{b}_1 =\bm{a}_1 \\ \\
& \bm{b}_2 =\bm{a}_2-\frac{(\bm{a}_2 ,\bm{b}_1)}{(\bm{b}_1 ,\bm{b}_1 )}\bm{b_1} \\ \\
& \bm{b}_3 =\bm{a}_3-\frac{(\bm{a}_3 ,\bm{b}_1)}{(\bm{b}_1 ,\bm{b}_1 )}\bm{b_1} 
\end{eqnarray*}
この式からそれぞれ$\bm{b}_1, \bm{b}_2, \bm{b}_3 を求めて,それぞれ大きさ1のベクトルにすれば正規直交基底が得られる.$
%
%
%
\newpage
%
%
%
\subsection{p124}
\noindent
問:[6.5]\\
\textsl{Hint}:直交補空間の定義から明らか,としては勿論いけない.直交補空間の定義から,地道に論証していく必要がある.\\ 
\dotfill \\
1)$^{\forall}\bm{x} \in \bm{W}に対し,(\bm{x},\bm{y})=0となる\bm{y} \in \bm{W}^{\perp}が存在する.よって(\bm{x},\bm{y})=(\bm{y},\bm{x})=0なので,\bm{y} \in \bm{W}^{\perp}に対し,\bm{x} \in (\bm{W}^{\perp})^{\perp}である.$\\
よって$\bm{x} \in W \Longrightarrow \bm{x} \in  (\bm{W}^{\perp})^{\perp}が言えたので,\bm{W} \subseteq  (\bm{W}^{\perp})^{\perp}である.$\\
また定理[4.7]から
\begin{eqnarray*}
& \dim \bm{W} +\dim \bm{W}^{\perp} =\dim (\bm{W} +\bm{W}^{\perp} )+\dim (\bm{W} \cap \bm{W}^{\perp} ) \\
& \dim \bm{W} +\dim \bm{W}^{\perp} =n
\end{eqnarray*}
$定理6.4から\mathbb{R}^nの計量空間\bm{V}は\bm{W}\dot{+}\bm{W}^{\perp}と表されること,[4.8]から,この直和の共通部分は\{ \bm{o} \}のみであることを用いた.$\\
また,同様に
\begin{eqnarray*}
\dim \bm{W}^{\perp} +\dim (\bm{W}^{\perp})^{\perp} =n
\end{eqnarray*}
でもあるので
\begin{eqnarray*}
\dim \bm{W} =\dim (\bm{W}^{\perp})^{\perp}
\end{eqnarray*}
よって$\bm{W} \subseteq  (\bm{W}^{\perp})^{\perp}と上の等式から,\bm{W} =  (\bm{W}^{\perp})^{\perp}$である.
%
%
%
\newpage
%
%
%
\subsection{p127-130}
\noindent
問1
\noindent \\
\textsl{Hint}:最も基本的な問題の1つで,共通部分を求めるとき,この場合は$s,t,u,v \in \mathbb{R}として,s\bm{a}_1+t\bm{a}_2=u\bm{a}_3+v\bm{a}_4 とおけば,この式を満たすs,t,u,vがどういうものなのかが必然と分かってくる.なお,問題にされていないが和空間の基底を求める場合は,\bm{a}_1,\bm{a}_2,\bm{a}_3,\bm{a}_4から線型独立になるようなベクトルの組を求めればいい.$\\
\dotfill \\
$s,t,u,v \in \mathbb{R}とすると$
\begin{eqnarray*}
s\bm{a}_1+t\bm{a}_2=u\bm{a}_3+v\bm{a}_4
\end{eqnarray*}
$とおけば,s,t,u,vの関係が導ける.この解空間は$
\begin{eqnarray*}
\begin{pmatrix}
1 & -1 & 0 & 1 \\
2 & 1 & -1 & 9 \\
0 & 3 & 5 & 1 \\
4 & -3 & 2 & 4 \\
\end{pmatrix}
\begin{pmatrix}
s \\
t \\
u \\
v \\
\end{pmatrix}
=\bm{o} \\
%%%
\begin{pmatrix}
1 & 0 & 0 & 3 \\
0 & 1 & 0 & 2 \\
0 & 0 & 1 & -1 \\
0 & 0 & 0 & 0 \\
\end{pmatrix}
\begin{pmatrix}
s \\
t \\
u \\
v \\
\end{pmatrix}
=\bm{o} \\
%%%
\begin{pmatrix}
s \\
t \\
u \\
v \\
\end{pmatrix}
=a
\begin{pmatrix}
-3 \\
-2 \\
1 \\
1 \\
\end{pmatrix}
\end{eqnarray*}
















%
%
%
\newpage
%
%
%
問2
\noindent





%
%
%
\newpage
%
%
%
\noindent
問3:











%
%
%
\newpage
%
%
%
\noindent
問4:











%
%
%
\newpage
%
%
%
\noindent
問5:







%
%
%
\newpage
%
%
%
\noindent
問6:





















%
%
%
\newpage
%
%
%
\noindent
問7:



















%
%
%
\newpage
%
%
%
\noindent
問8:





























%
%
%
\newpage
%
%
%
\noindent
問9:





























%
%
%
\newpage
%
%
%
\noindent
問10:




















%
%
%
\newpage
%
%
%
\noindent
問11:





















%
%
%
\newpage
%
%
%
\noindent
問12:

















%
%
%
\newpage
%
%
%
問13:














%
%
%
\newpage
%
%
%
\section{第5章:固有値と固有ベクトル(解答)}
%
%
%
\newpage
%
%
%
\section{第6章:単因子およびジョルダンの標準形(解答)}
%
%
%
\newpage
%
%
%
\section{第7章:ベクトルおよび行列の解析的取扱い(解答)}

%
%
%
\newpage
%
%
%
\section{附録III~群および体の公理}
\subsection{p249 体の公理}
問:\\
イ)
\begin{leftbar}
	\begin{proof}
		体$K$の単位元について,$0=0+0$であるから,
		\begin{gather*}
			a 0=a(0+0)=a0 + a0\\
			\therefore \quad  a0 = a0 + a0
		\end{gather*}
		$K$は加法について可換群であるから,$a0$の逆元$-a0$が$K$に存在する.これを用いると,
\begin{gather*}
	a0 + (-a0) = a0 + a0 + (-a0) \\
	\therefore \quad 0 = a0 + a0 +(-a0)
\end{gather*}
 ここで,
 \begin{align*}
	a0 + a0 +(-a0)&=a0+ \{a0+(-a0)\} \\
	& = a0 + 0 \\
	& = a0
 \end{align*}
となるから,$0=a0$である.$0=0a$についても同様.
\end{proof}
\end{leftbar}
(ロ)
\begin{leftbar}
    \begin{proof}
        $a \ne 0$とする.このとき,$a$の逆元$a^{-1} \in K$が存在し,$ab=0$の両辺に$a^{-1}$をかけると,
        \begin{gather*}
            a^{-1} (ab) = a{-1} 0 \\
            (a^{-1}a)b =0 \\
            1b =0 \\
            \therefore b=0
        \end{gather*}
        である.これと$b \ne 0$を仮定したときの同様の考察により,$ab=0$のとき,$a=0$または$b=0$である.
    \end{proof}
\end{leftbar}

\newpage 

\setcounter{equation}{0}
\subsection{p255~256}

問7:

\begin{leftbar}
    \begin{proof}
        $M=\{x>0 \mid x^k \le a\}$とおく.$M$は上に有界であり,$M \ne \varnothing$であるから,ワイエルシュトラスの公理により,$\alpha = \sup M$が存在する.\par 
        (1) \quad $\alpha^k \le a$であることを示す.$\alpha^k >a$と仮定すると,$\varepsilon >0$を十分小さく取ったとき,$(\alpha - \varepsilon)^k >a$とできるが,
        上限の定義から,$\alpha - \varepsilon <x$となる$x \in M$が存在し,このとき$x^k >a$となり$M$の定義に矛盾.\par 
        (2) \quad $ \alpha^k \ge a$であることを示す.$\alpha^k <a$と仮定すると,$\varepsilon>0$を十分小さく取ったとき,
        $(\alpha + \varepsilon)^k<a$となり,$M$の定義から$\alpha + \varepsilon \in M$となるが,これは$\alpha = \sup M$であることに矛盾する.\par 
        (1),(2)より,$\alpha^k =a$となり,$x^k=a$となる$x$の存在が示された.\par 
        一意性を示す.${\alpha_1}^k =a$なる$\alpha_1>0$の存在を仮定すると,
        \begin{gather*}
            (\alpha - \alpha_1)(\alpha^{k-1}+{}_n \mathrm{C} _1 \alpha^{k-2} \alpha_1 + \cdots +{}_n \mathrm{C} _{n-1} \alpha {\alpha_1}^{k-2} + {\alpha_1}^{k-1})= \alpha^k -{\alpha_1}^k =a-a=0 \\
            \therefore \quad \alpha = \alpha_1 \quad (\because~\alpha^{k-1}+{}_n \mathrm{C} _1 \alpha^{k-2} \alpha_1 + \cdots +{}_n \mathrm{C} _{n-1} \alpha {\alpha_1}^{k-2} + {\alpha_1}^{k-1}>0 )
        \end{gather*}
        これより一意性も示された.
        \end{proof}
    \end{leftbar}









\end{document}